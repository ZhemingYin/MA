\begin{minipage}[t]{\textwidth}
    \chapter*{Abstract}
    \addcontentsline{toc}{chapter}{Abstract}
During the extensive usage of chirp sequence radar, it is often affected by the cost, system limit or regulations, resulting in limited resolution of the range-Doppler map. Meanwhile, the range-Doppler map collected in the indoor environment also puts higher requirements on the model in terms of complexity and high amplitude fluctuation, whereas the current upsampling approaches cannot meet this requirement well. Therefore, this thesis will combine the current upsampling models in the state of the art, such as Transformer and cGAN models, with a series of data processing methods, such as logarithm and normalization operations, and the combination of multiple loss functions to improve the quality of the super-resolution range-Doppler map. In addition, the environmental conditions and data in the public datasets about the range-Doppler map are currently simple, so the dataset of this thesis is collected in the indoor environment by ourselves. In the evaluation phase, we obtain an optimized combination of model, processing methods and loss functions and tune the hyperparameters of the model, which significantly improves the result compared to the other image upsampling approaches.
\end{minipage}%
\hfill
\begin{minipage}[t]{\textwidth}
    \chapter*{Kurzfassung}
    % \addcontentsline{toc}{chapter}{Kurzfassung}
Bei der umfangreichen Nutzung von Chirp-Sequence-Radar ist es häufig von den Kosten, Systembeschränkungen oder Vorschriften beeinflusst, was zu einer limitierten Auflösung der Range-Doppler-Karte führt. Gleichzeitig stellt die in Innenräumen erfasste Range-Doppler-Karte höhere Anforderungen an das Modell hinsichtlich Komplexität und hoher Amplitudenfluktuation, denen aktuelle Upsampling-Methoden nicht gut gerecht werden können. Daher kombiniert diese Masterarbeit aktuelle fortgeschrittene Upsampling-Modelle, wie Transformer- und cGAN-Modelle, mit einer Reihe von Datenverarbeitungsmethoden wie Logarithmierung und Normalisierung sowie die Kombination mehrerer Verlustfunktionen, um die Qualit\"at der hochauflösenden Range-Doppler-Karte zu verbessern. Da die Umgebungsbedingungen und Daten in öffentlichen Datens\"atzen zur Range-Doppler-Karte derzeit noch einfach sind, wurde der Datensatz dieser Arbeit selbst in Innenräumen erfasst. In der Evaluierungsphase erzielen wir eine optimierte Kombination aus Modell, Verarbeitungsmethoden und Verlustfunktionen und optimieren die Hyperparameter des Modells, was das Ergebnis im Vergleich zu anderen Bild-Upsampling-Methoden deutlich verbessert.
\end{minipage}